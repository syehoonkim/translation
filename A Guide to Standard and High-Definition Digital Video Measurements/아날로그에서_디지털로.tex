\chapter{아날로그에서 디지털로}
디지털 데이터 스트림은 각각의 개별 구성 요소들로 쉽게 분해될 수 있는데, 이들은 보통 아날로그에서 대응되는 것들과 같은 역할을 한다.
아날로그와 디지털 비디오 영역을 설명하고 비교하는 동안 계속 이 관계를 사용할 것이다.
한번 아날로그와 디지털 비디오의 유사성을 이해하고 나면 HDTV를 다룰 수 있는데, 보통 이는 이에 대응되는 HD 아날로그 포맷의 디지털적 표현이다.
\\
NTSC와 PAL 비디오 신호들은 색의 삼원색인 빨강, 초록, 파랑 세 개의 카메라 채널의 합성인데, 행렬을 이용하여 합쳐져서 휘도를 만들고 이는 두 개의 색상 정보를 담고 있는 반송파 억압 변조의 결과물과 합쳐진다.
세 번쨰 단일 채널 컴포지트 전송 시스템은 SECAM 시스템인데, 이는 색상 정보를 전달하기 위해 한 쌍의 주파수 변조된 부반송파들을 이용한다.
스튜디오에서는 카메라의 RGB 감지 장치와 종단 디스플레이의 RGB 채널 사이의 어느 단에서도 신호가 NTSC, PAL 혹은 SECAM이 되어야 할 특별한 요구 조건은 없다.
NTSC, PAL 또는 SECAM에 대한 이해가 충분히 유용한 이상, 컴포지트 비디오에 대해서 더 많은 이해를 위해 투자하진 않을 것이다.

\section{RGB 컴포넌트 신호}
비디오 카메라는 이미지를 빛의 삼원색인 빨강, 초록, 파랑으로 분할한다. 카메라의 센서들은 이 각각의 단색 이미지들을 분리된 전기 신호로 변환한다.
그림의 왼쪽 끝과 최상단을 알려주는 동기 정보가 이 신호들에 추가된다. 디스플레이를 카메라와 동기화시키는 정보가 초록 채널에, 때로는 모든 채널에 더해지거나 아니면 별도로 전달된다.
\\
가장 간단한 배선은 {그림 1}에 나와 있듯이 R, G, B를 카메라에서 그대로 뽑아서 모니터에 연결하는 것이다. 여러 선을 이용한 전송 시스템은 아날로그 SD에서나 HD 비디오에서나 같다.
여러 선을 이용한 연결은 작고, 영구적으로 구성된 부분 시스템에서 사용될 수 있을 것이다.
\\
이 방법은 카메라에서 디스플레이까지 고품질의 이미지를 만들어내지만, 신호를 세 분리된 채널로 전송하기 위해선 엔지니어가 각 채널이 신호를 처리할 때 같은 이득, 직류 오프셋, 시간 딜레이와 주파수 응답을 갖게 해야 한다.
각 채널의 이득이 다르거나 직류 오프셋에 오차가 생기면 최종 디스플레이 출력에서 미묘한 색상 변화가 일어날 것이다.
시스템에 타이밍 오차가 있을 수도 있는데, 이는 케이블 길이가 다르거나 각 신호를 카메라에서 디스플레이까지 라우팅하는 경로가 달라서 생길 수 있다.
이는 채널간 타이밍 오프셋을 만들 것이고 영상이 뭉개지게 만들 것이고, 심한 경우 이미지가 분리되어 여러 개가 나타날 것이다.
주파수 응답의 차이는 채널이 합쳐질 때 일시적인 악영향을 만들 것이다.
분명히 세 채널을 하나로 다룰 방법이 필요하다.
\\
NTSC나 PAL 인코더와 디코드를 {그림 2}와 같이 추가하는 것은 방송국 내에서 신호가 하나의 선으로 다뤄지는 것 외에는 단순화에 도움을 주지 못한다.
시스템 대역폭은 세 비디오 신호의 에너지를 4.2 MHz(NTSC)나 5.0에서 5.5 MHz(PAL) 내에서 다루기 적절하게 정해진다.
단일 선 구성은 비디오 라우팅을 쉽게 해 주지만, 더 먼 경로에 대해서 주파수 응답과 타이밍 문제를 고려해야 한다. 