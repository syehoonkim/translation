\chapter{타이밍과 동기화}
표준은 비디오가 제작되는 처음과 끝 전체에 대해 다양한 장비들 간의 상호 운용성과 교환성을 가능하게 하는 정보를 제공한다. 좋은 표준은 자원과 기술을 경제적으로 이용할 수 있게 한다.
표준은 사용자들 간의 협동과 혁신이 일어나기 좋게 한다. 표준은 영상 전문가들이 제작하는 프로그램이 가정 시청자들에게 동일하게 보이게 하기 위해서 필수적이다.

미 국립 표준 협회(ANSI), 동영상 및 텔레비전 엔지니어 협회(SMPTE), 오디오 엔지니어링 협회(AES) 및 국제 통신 연합(ITU)는 영상과 오디오에 대한 참조 표준과 권고안을 발행한다.
부록 D - 텔레비전에 대한 참고 표준 목록에 있는 대표 표준과 권고안들은 호환성과 규제 준수를 위한 신호 파라미터들을 정의하고 있다.
이러한 단체들은 표준을 매우 신중하게 만들고, 이렇게 만들어진 표준들은 각 시스템들을 정확히 설명해준다. 뒤이어 나올 논의들은 각각 표준화된 다양한 포맷들에 대한 일반적인 이해를 제공하기 위한 각 표준들을 해석한 것이다.

비디오 화면을 성공적으로 제작, 전송하고 수신하여 복원하기 위해서는 시스템의 각 장치들이 서로 동기화되어 작동해야 한다. 카메라가 감지한 장면의 특정 요소는 수상기 화면의 어디에 그 요소가 복원되어 보여야 할지 판정할 방법이 필요하다.
동기화 요소는 여러 카메라들과 영상 소스에게 어떻게 함께 영상을 만들지, 그리고 영상이 최종단에서 표현될 때 수상기가 어떻게, 그리고 어느 위치에 영상 신호를 복원해야 할지 알려준다.

카메라와 최종단 디스플레이는 촬상면과 스크린을 어떻게 주사해야 하는지 알고 있다. 필요한 것은 어디에서 시작하고 어떻게 단계적으로 주사해나갈 것인지이다. 동기 정보는 각 수평 줄과 디스플레이 전체에 대한 각 수직 스윕마다(2:1 비월 주사 포맷의 경우 전체 영상에 대한 두 스윕마다) 나타난다.
대형 스튜디오 시설에서는 별도의 주 동기신호 발생기가 동기 정보를 제공한다. 작은 시스템에서는 하나의 카메라가 기준이 되어 자신과 다른 영상 소스들에 동기 신호를 제공할 수도 있다.

\section{아날로그 비디오 타이밍}
흔히 사용되는 아날로그 비디오 포맷 표준은 다음 6개가 있다: PAL, PAL-M, PAL-N, 셋업된 NTSC, 셋업 없는 NTSC, SECAM. 여기에 추가로 어떤 나라들에서는 온에서 송신 대역폭을 좀 더 넓게 하여 보다 넓은 영상 대역폭을 위한 여유를 둔다.
SECAM을 사용하는 국가들의 스튜디오에서는 컴포넌트나 PAL로 영상을 제작한 다음 송신 시에만 SECAM 포맷으로 한다. SECAM과 PAL은 색상 신호가 휘도 영상에 어떻게 변조되어 더해지는지를 제외하면 비슷하다.

스튜디오 영상은 발생하자 마자 사용되거나, 다른 영상 소스와 맞춰지기 위해 기다려지거나, 아니면 추후 재생을 위해 녹화되는 연속적인 정보의 스트림이다. 전달될 때에는 실시간으로 전달되며, 목적지에서 영상으로 복원되기 위해 필요한 정보들과 함께 전달되어야 한다.
영상은 화면 정보와 타이밍 정보를 포함하여 화면을 재생할 수 있도록 한다. 타이밍 정보는 영상의 각 줄을 나타내는 주기적으로 발생하는 수평 동기 펄스나 정해진 데이터 워드들을 포함하고, 여기에 디스플레이가 최상단부터 화면을 그리도록 알려주는, 덜 자주 발생하는 수직 동기 정보가 끼어든다.

NTSC나 PAL 컴포지트 영상 포맷에서는 영상과 타이밍 정보가 쉽게 관찰된다. 영상 웨이브폼 모니터는 영상의 수평 줄, 수평 블랭킹 구간, 영상 전체의 수평 줄들에 대한 스윕, 수직 블랭킹 구간 내의 줄들을 디스플레이하기 위한 스윕 속도 프리셋들을 갖고 있다.
중요한 것은 모두 똑같은 영상 신호에 대한 디스플레이이며, 다른 것은 신호가 언제, 얼마나 오래 표시되냐는 것이다. 현대적인 용어로는 컴포지트 아날로그 비디오는 휘도 영상과 동기 정보가 시분할 다중화된 것이다.
그리고 색상 정보는 두 개의 색차 채널이 주파수 분할 다중화된 것이다.

\section{수평 타이밍}
\image{!t}{fig18 NTSC horizontal blanking interval.pdf}{NTSC 수평 블랭킹 구간}{fig18 NTSC horizontal blanking interval}
\image{!h}{fig19 PAL horizontal blanking interval.pdf}{NTSC 수평 블랭킹 구간}{fig19 PAL horizontal blanking interval}
525/59.94 NTSC 주사 방식의 수평 타이밍 그림(\figurename~\ref{fig18 NTSC horizontal blanking interval}) 및 625/50 PAL 주사 방식에 대한 그림(\figurename~\ref{fig19 PAL horizontal blanking interval})\을 보면 둘은 개념적으로 비슷하고 1900년대 중반 당시의 카메라와 디스플레이 장비들의 제약 사항을 고려하여 개발되었다.
수평 블랭킹 구간은 비디오 정보의 줄당 한번 나타나고 수직 블랭킹 구간을 위해 조정된다.

수평 프론트 포치(역주: FRONT PORCH를 직역하면 앞 현관이다)는