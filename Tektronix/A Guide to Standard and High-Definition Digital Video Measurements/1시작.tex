\chapter{시작}
디지털 텔레비전은 뭔가 매우 과학적이고 복잡한 것이라고 생각하기 쉽다. 하지만 최종 결과물은 친숙한 것이다.
그것은 바로 텔레비전 엔지니어들이 처음부터 계속 추구해온 것들, 즉 계속해서 좋아지는 시청 경험, 바로 아티스트의 퍼포먼스를 시청자들에게 더 잘 전달하는 양질의 비디오와 오디오이다.
디지털 텔레비전에서 유일하게 새로운 것은 정보(메시지)가 전달되는 방법 뿐이다.


메시지가 어떻게 전달되는지가 정말 중요할까? 아티스트와 시청자들(그리고 많은 나라에서 광고주들)은 신호가 전달되는 경로를 신경쓰지 않을 것이다.
그들은 디지털 텔레비전의 개선된 품질만을 누릴 뿐이지 자세한 것은 모른다. 하지만 이론, 바로 이것이 재미있는 부분이다.
텔레비전의 기술적인 부분에 관련있는 우리 엔지니어들은 이론에 신경을 쓰고, 지난 60년 이상의 시간 동안 크게 발전한 텔레비전 이론의 혜택을 누린다.
그 중에서 지난 20년간 디지털 텔레비전이 가져온 발전을 특히 더 누린다.


프로그램 영상, 디지털 오디오, 그리고 관련된 부가 데이터 신호들이 모여서 디지털 텔레비전 신호를 이룬다.
아날로그 텔레비전 세상에서는 비디오와 오디오는 완전히 분리된 경로를 통해 소스로부터 가정의 수상기로 전달된다.
디지털 신호들은 훨씬 더 자유롭게 비디오, 오디오 및 다른 신호들이 하나의 데이터 흐름으로 엮여서 구성될 수 있다.
우리가 알아야 할 것은 데이터가 어떻게 구성되어서 우리가 원하는 것을 어떻게 골라낼 수 있는지 뿐이다.

\section{전통적인 텔레비전}
아날로그 비디오와 아날로그 오디오를 전통적인 텔레비전의 요소라고 할 수 있을 것이다. 하지만 우리는 전통적인 목표를 이루기 위해, 어쩌면 더 많은 것을 이루기 위해 노력한다는 것을 알아야 한다.
디지털 텔레비전은 아날로그에 기반하고, 디지털 텔레비전에 대한 이해는 아날로그 텔레비전의 이해에서 나온다.
카메라 렌즈로 들어가는 빛과 마이크로 들어는 소리는 여전히 아날로그이다. 디스플레이에서 빛이 나오고 소리가 귀로 들어가는 것 역시 아날로그 현상이다.


우리는 이미 아날로그 비디오는 빛의 값을 샘플링한 것임을 알고 있다. 밝기 값은 전압으로 표현된다. 거기에 추가적인 정보가 샘플의 색상을 알려준다.
샘플들은 동기화된 상태로 전송 시스템 내에서 전달되어 디스플레이에서 원래 이미지를 다시 만들어내게 된다.
아날로그 비디오는 수상기가 적절히 처리하는 방법을 알면 그림을 다시 만들어낼 수 있는 데이터들을 담고 있는 전압값들의 직렬 흐름으로서 전달된다.
따라서 단순히 몇몇 단어만 바꾸고, 지난 50년간 배워온 것들을 이용하기 위해서 몇몇 가지만 다르게 하면 디지털 비디오는 아날로그 비디오와 그리 다르지 않음을 알 수 있다.


그럼, 아날로그 빛에서 시작해서 아날로그 빛으로 끝난다면 디지털 비디오를 도대체 왜 쓰는 것일까?
많은 경우, 카메라 센서는 아날로그 비디오를 만들어내지만, 거의 즉시 순간순간의 영상의 값을 나타내며 변동하는 아날로그 전압값을 디지털로 바꿔서 근본적으로 열화 없이 다룰 수 있게 한다.
컴퓨터 그래픽과 같은 몇몇 경우에는 비디오가 디지털로 시작하며, 최신의 디지털 텔레비전 시스템 내에서는 아날로그로 변환되지 않고 디스플레이로 도달한다.


우리는 텔레비전 신호를 여전히 아날로그 NTSC, PAL, SECAM 전송 방식으로 보낼 수 있지만, 더 높은 품질과 더 높은 효율로 텔레비전 신호를 보내기 위해서 디지털 전송을 하고 있다.
디지털 텔레비전은 일상 생활내에서 누릴 수 있는 일부가 되었다. 누군가는 이를 이용하며 발전에 기여할 것이고, 누군가는 세부사항을 모른 채 혜택을 누릴 것이다.