\chapter{``새로운'' 디지털 텔레비전}
디지털 신호는 오랜 신간 동안 텔레비전의 일부가 되어왔다. 처음에는 테스트 신호와 자막 생성기 같은 장비 내에서 보이지 않게 묻혀있다가, 나중에는 전체 시스템으로 확산되었다.
이 글에서는 이해를 쉽게 하기 위해 위해 우선 텔레비전 신호의 비디오 부분을 다룰 것이다.
오디오 또한 마찬가지로 디지털일 것이고, 디지털 데이터가 복원되는 텔레비전 수상기에서 재생될 것이다. 디지털 오디오는 뒤의 장들에서 다뤄질 것이다.

디지털 비디오는 아날로그 비디오를 간단히 확장한 것이다. 한번 아날로그 비디오를 이해하고 나면, 디지털 비디오가 어떻게 만들어져서 다뤄지고, 처리되고, 아날로그 신호가 되고, 아날로그 신호에서 변환되는지 이해하기 쉽다.
아날로그와 디지털 비디오에는 비슷한 제약 조건들이 많이 있고, 디지털 영역에서 발생할 수 있는 많은 문제들은 정확하지 않은 아날로그 소스 비디오에 기인한다.
따라서, 아날로그와 디지털 비디오 장치의 설계와 작동에 대한 기준이 되는 표준을 마련하는 것이 중요하다.

\section{아날로그 세계를 설명하는 숫자들}
초기 디지털 비디오는 단순히 아날로그 NTSC나 PAL 컴포지트 비디오 신호의 디지털 표현일 뿐이었다.
표준들은 작동 한계를 설명하고 각 전압 레벨을 설명하는 숫자들과 각 숫자들이 어떻게 만들어지고 복원되는지를 규정했다.
데이터 속도가 빨랐기 때문에 디지털 비디오 데이터를 내부적으로 8비트나 10비트 버스로 다루는 게 일반적이었고, 마찬가지로 초기 표준들은 여러 선을 이용한 외부 연결을 다뤘다.
표준들은 수상기와 전달된 데이터들을 동기화하고 또 임베디드 오디오와 같은 추가적인 기능을 제공하는 특정한 부가 데이터와 관리 데이터들을 설명했다.
후에 고속 처리가 가능해지면서 하나의 선을 이용한 직렬 인터페이스 표준이 만들어졌다.
기본적으로, 디지털 비디오는 아날로그 전압의 숫자 표현이며, 빠르게 변하는 비디오와 필요한 부가 데이터를 담기 충분하게 빠르게 나타나는 숫자 데이터들이다.

\section{컴포넌트 디지털 비디오}
초기 아날로그 효과 장치 설계자들은 빨강, 초록, 파랑 채널을 신호처리 과정에서 최대한 분리하는 것의 이점을 깨달았다.
NTSC와 PAL 인코딩/디코딩 과정은 투명하지 않으므로(역주: 정보를 그대로 전달하지 않으므로) 여러 번 인코딩하고 디코딩하는 것은 점점 신호를 열화시킨다.
카메라 신호는 빨강, 초록, 파랑 각각의 독립적인 정보 채널에서 시작하고, 이 신호들을 시스템 내에서 최대한 적게 포맷 변환을 해서 NTSC나 PAL 신호로 인코딩하여 시청자들에게 보내는 게 가장 좋다.
하지만 이 세 개의 독립적인 좌표축으로 나타나는 정보 채널들을 방송국 내에서 다루는 것은 물류 문제(역주: 케이블 포설 등)와 신뢰성 문제를 일으킨다.
실질적인 관점에서, 이 세 신호는 하나의 전선, 보통 하나의 동축 케이블 내에 같이 존재해야 한다.
이 세 가지 빨강, 초록, 파랑 비디오 채널들을 행렬을 이용하여 섞어서 보다 효율적인 휘도와 두 가지 색차 신호 집합으로 만들어서 각각을 디지털화한 후 하나의 동축 케이블에 다중화해서 담으면 된다.
이렇게 하면 이 데이터 신호를 NTSC나 PAL 컴포지트 비디오 다루듯이 다룰 수 있다. 이제 고속의 숫자 데이터 흐름을 다루게 된다.
이 데이터 신호는 NTSC나 PAL 비디오 신호의 5$\sim$6 MHz 신호보다 훨씬 빨리 변하는 에너지를 담고 있지만, 어느 정도 실질적인 거리까지는 별다른 조치 없이도 손실 없이 다룰 수 있다.
한번 비디오 신호가 디지털화되면, 우리는 디지털 영역 내에서 추가적인 손실이나 채널 간 영향 없이 각 구성 요소들을 추출해서 개별적으로 처리하고 다시 합칠 수 있다.

디지털의 각 요소와 기술은 비디오 품질 관리에 큰 도움을 주었으며, 디지털 장치들의 빠른 속도는 HD 비디오가 요구하는 넓은 대역폭을 실현 가능하게 했다.
디지털을 통해서 필요한 데이터량을 줄이는 다양한 압축 알고리즘을 이용한 처리도 가능하다.
따라서 HD 비디오와 다채널 오디오를 고품질 실시간 아날로그 비디오의 대역폭 내에서 전송할 수 있다.
비디오 압축이라는 주제는 많은 출판물에서 다뤄지고 있으며(참고 문헌을 보라) 이 글에서는 다루지 않을 것이다.